%%%%%%%%%%%%%%%%%%%%%%%%%%%%%%%%%%%%%%%%%
% Arsclassica Article
% LaTeX Template
% Version 1.1 (10/6/14)
%
% This template has been downloaded from:
% http://www.LaTeXTemplates.com
%
% Original author:
% Lorenzo Pantieri (http://www.lorenzopantieri.net) with extensive modifications by:
% Vel (vel@latextemplates.com)
%
% License:
% CC BY-NC-SA 3.0 (http://creativecommons.org/licenses/by-nc-sa/3.0/)
%
%%%%%%%%%%%%%%%%%%%%%%%%%%%%%%%%%%%%%%%%%

%----------------------------------------------------------------------------------------
%	PACKAGES AND OTHER DOCUMENT CONFIGURATIONS
%----------------------------------------------------------------------------------------

\documentclass[
10pt, % Main document font size
a4paper, % Paper type, use 'letterpaper' for US Letter paper
oneside, % One page layout (no page indentation)
%twoside, % Two page layout (page indentation for binding and different headers)
headinclude,footinclude, % Extra spacing for the header and footer
BCOR5mm, % Binding correction
]{scrartcl}

\input{structure.tex} % Include the structure.tex file which specified the document structure and layout

\hyphenation{Fortran hy-phen-ation} % Specify custom hyphenation points in words with dashes where you would like hyphenation to occur, or alternatively, don't put any dashes in a word to stop hyphenation altogether

%----------------------------------------------------------------------------------------
%	TITLE AND AUTHOR(S)
%----------------------------------------------------------------------------------------

\title{\normalfont\spacedallcaps{JMATRW User Manual}}

\author{\spacedlowsmallcaps{JMATRW v0.2 beta}}

%\author{\spacedlowsmallcaps{John Smith* \& James Smith\textsuperscript{1}}} % The article author(s) - author affiliations need to be specified in the AUTHOR AFFILIATIONS block

%\date{} % An optional date to appear under the author(s)

%----------------------------------------------------------------------------------------

\begin{document}

%----------------------------------------------------------------------------------------
%	HEADERS
%----------------------------------------------------------------------------------------

\renewcommand{\sectionmark}[1]{\markright{\spacedlowsmallcaps{#1}}} % The header for all pages (oneside) or for even pages (twoside)
%\renewcommand{\subsectionmark}[1]{\markright{\thesubsection~#1}} % Uncomment when using the twoside option - this modifies the header on odd pages
\lehead{\mbox{\llap{\small\thepage\kern1em\color{halfgray} \vline}\color{halfgray}\hspace{0.5em}\rightmark\hfil}} % The header style

\pagestyle{scrheadings} % Enable the headers specified in this block

%----------------------------------------------------------------------------------------
%	TABLE OF CONTENTS & LISTS OF FIGURES AND TABLES
%----------------------------------------------------------------------------------------

\maketitle % Print the title/author/date block

\setcounter{tocdepth}{2} % Set the depth of the table of contents to show sections and subsections only

\tableofcontents % Print the table of contents

%\listoffigures % Print the list of figures

%\listoftables % Print the list of tables

%----------------------------------------------------------------------------------------
%	ABSTRACT
%----------------------------------------------------------------------------------------

%\section*{Abstract} % This section will not appear in the table of contents due to the star (\section*)

%\lipsum[1] % Dummy text

%----------------------------------------------------------------------------------------
%	AUTHOR AFFILIATIONS
%----------------------------------------------------------------------------------------

%{\let\thefootnote\relax\footnotetext{* \textit{Department of Biology, University of Examples, London, United Kingdom}}}

%{\let\thefootnote\relax\footnotetext{\textsuperscript{1} \textit{Department of Chemistry, University of Examples, London, United Kingdom}}}

%----------------------------------------------------------------------------------------

\section{Introduction}

The basic JMATRW is a Java library to read mat files, mathematical files with extension *.mat generated for instance with the software GNU Octave\footnote{GNU Octave web-site: \url{https://www.gnu.org/software/octave/index.html}}. This version, the library is able to read arrays of doubles. 
  
\section{JMATRW}

The following example shows how to read a mat file with an array in Java.

\begin{lstlisting}[style=myJavaStyle]
//Open the file.
InputStream fis = new FileInputStream("/tc01NoCompression/example01_array.mat");
		
//Create the reader.
JMATReader reader = new JMATReader(fis);
JMATInfo matdata = reader.getInfo();
		
//Some JMATInfo properties.
boolean bIsArray = (matdata.dataType == DataType.ARRAY_DOUBLE);
String sArrayName = matdata.dataName;
	
//It counts and sums the array values.
int count = 0;
double sum = 0;
while (reader.hasNext()) {
	double value = reader.next();
	sum += value;
	count++;
}
\end{lstlisting}

\section{JMATRW for Apache Spark}

JMATRW is able to read huge arrays of doubles (and in the next releases matrix of doubles). JMATRW4Spark enables the reading of mat data in Apache Spark. In this way, huge arrays can be read and processed using High Performance Computing (HPC) resources.

\subsection{Scala example}

The following example shows how to use the library within the Spark-shell.

\begin{lstlisting}[style=myScalaStyle]
import java.lang.Long
import java.lang.Double
import it.prz.jmatrw4spark.JMATFileInputFormat
import org.apache.spark.SparkContext._

import org.apache.hadoop.conf.Configuration
val conf = new Configuration(sc.hadoopConfiguration)

val matrdd = sc.newAPIHadoopFile("vecRow03_x256.mat", classOf[JMATFileInputFormat], classOf[Long], classOf[Double], conf)

val matvalues = matrdd.map(x => x._2.toDouble)
val stat = matvalues.stats()
\end{lstlisting}

JMATRW4Spark provides an InputFormat \verb|JMATFileInputFormat| to read mat files and split their content in multiple pieces. After the split phase, each obtained block refers to a part of the input array, which is assigned to a Apache Spark Worker. The library provides also a custom RecordReader to independently read specific parts of the a mat file. In this way, after the split and the block assignment, each work can read its assigned data (array items) and process them.

\begin{quote}
NOTE on compressed mat files - JMATRW reads also compressed mat files, but they are not split. So, exactly one worker reads the entire mat file without split. In order to avoid this, be sure to save mat files compatible with mat version 5 without compression.
\end{quote}

\subsection{Install JMATRW on Apache Spark 1.5.2}

In order to install the JMATRW4Spark library under Spark 1.5.2, prepare a directory (e.g., /libthirdparty) coping the jar files in it. Open the file \verb|conf/spark-defaults.conf| and past these lines in it, changing also the paths provided as example.

\begin{verbatim}
spark.driver.extraClassPath YOUR_SPARK_INST_DIR/libthirdparty/*
spark.executor.extraClassPath YOUR_SPARK_INST_DIR/libthirdparty/*
\end{verbatim}

\section{License}

JMATRW Library and this documentation is released under the LGPL v3 license. 

%\newpage % Start the article content on the second page, remove this if you have a longer abstract that goes onto the second page


%----------------------------------------------------------------------------------------
%	BIBLIOGRAPHY
%----------------------------------------------------------------------------------------

%\renewcommand{\refname}{\spacedlowsmallcaps{References}} % For modifying the bibliography heading

%\bibliographystyle{unsrt}

%\bibliography{sample.bib} % The file containing the bibliography

%----------------------------------------------------------------------------------------

\end{document}